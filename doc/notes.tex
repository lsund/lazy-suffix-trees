\documentclass[a4paper]{article}

\usepackage[english]{babel}
\usepackage[utf8]{inputenc}
\usepackage{listings}
\usepackage{color}
\usepackage{hyperref}
\usepackage{float}
\usepackage{epigraph}
\usepackage{amsmath,amssymb}
\usepackage[toc,page]{appendix}
\usepackage{listings}
\usepackage{changepage}

\begin{document}


\section{Some Basic Preprocessor Definitions}

We need four space blocks. Their names remind of the pointers, by which
the blocks are referred to. The \texttt{stack}-block is only used for
eager evaluation.\\

Each branching node requires 2 unsigned integers. Hence the macro
\texttt{BRANCHWIDTH}. If additional information has to be stored for a
branching node, then \texttt{BRANCHWIDTH} must be redefined accordingly.\\

The suffix tree is stored in a table \texttt{stree}, called \(T\) in the
paper. Given a pointer into \texttt{stree}, we can determine the index the
node is stored at with \texttt{INDEX}\\

The most significant bits in the entries of table \texttt{stree} are used as
marking bits. For each branching node we need 3 marking bits, and for each leaf
we need 2 marking bits. For each leaf and each branching node we store 2
marking bits in their first integer: The \emph{leaf-bit} and the
\emph{rightmost-child} bit. For each branching node we store 1 marking bit in
the second integer, the \emph{unevaluated} bit.\\

As a consequence, we have 30 bits to store \(\mathit{left}(\overline{u})\) or
\(\mathit{lp}(\overline{u})\) for a branching node or leaf \(\overline{u}\).
Moreover, we have 31 bits to store \(\mathit{right}(\overline{u})\) or
\(\mathit{firstchild}(\overline{u})\) for a branching node \(\overline{u}\).
\(\mathit{left}(\overline{u})\), \(\mathit{right}(\overline{u})\), and
\(\mathit{lp}(\overline{u})\) are in the range \([0,n]\).
\(\mathit{firstchild}(\overline{u})\) is in the range \([0,3n]\). Hence
\(3n\leq 2^{31}-1\) must be satisfied, i.e.\ the maximal length of the input
string is \(715{,}827{,}882\)\\

The macros \texttt{ISLEAF, ISRIGHTMOSTCHILD, ISUNEVALUATED} test whether the
corresponding bits are set. The argument is the pointer to the \emph{first}
integer of the corresponding record.\\

Given a pointer to the first index of a branching or leaf record, we obtain the
\emph{lp}-value and the \emph{firstchild}-value by the macros
\texttt{GETLP,GETFIRSTCHILD}. There are the two macros
\texttt{SETLP,SETFIRSTCHILD} to set these values. Note that in
\texttt{GETFIRSTCHILD} we do not have to strip the \emph{unevaluated}-bit,
since this is not set for an unevaluated node. In contrast, in \texttt{SETLP}
we have to take care that the rightmost child is maintained. For a leaf we set
the \emph{lp}-value and the leaf bit.\\

References are unsigned integers. An undefined reference is
\texttt{UNDEFREFERENCE} constant.\\

We store suffix pointers in an array \texttt{suffixes}. For each suffix
pointer, we determine the corresponding suffix number with
\texttt{SUFFIXNUMBER}.\\

An unevaluated branching node \(\overline{u}\) is represented by a record of
two integers which store two pointers \(\mathit{left}(\overline{u})\) and
\(\mathit{right}(\overline{u})\). These pointers are relative to a pointer
\texttt{suffixbase}. The macros \texttt{STOREBOUNDARIES}
\texttt{GETLEFTBOUNDARY} \texttt{GETRIGHTBOUNDARY} are used for storing and
retrieving these pointers.\\

To retrieve the \emph{lp}-value of an unevaluated node, we first retrieve the
left boundary. This gives a pointer into the array \texttt{suffixes}. From the
pointer we can retrieve the corresponding \emph{lp}-value using
\texttt{GETPUNEVAL}

\section{Declaration of Global Variables}

  \texttt{rootchildtab} is initialized in \texttt{evalrootsuccedges}.
  For each ASCII-character \(a\) the following holds:

  \begin{itemize}
    \item
        If there is no \(a\)-successor, then
        \(rootchildtab[a]=\texttt{UNDEFINEDSUCC}\), where the latter constant is
        defined below.
    \item
        If the \(a\)-successor leads to the leaf stored at index \(i\), then
        \(rootchildtab[a]=i\texttt{ | LEAFBIT}\).
    \item
        If the \(a\)-successor leads to the branching node stored at index \(i\),
        then \(rootchildtab[a]=i\).
  \end{itemize}

\section{Space Management}

The function \texttt{getbufferspaceeager} does the space management for the
array \texttt{sbuffer} in the case of eager evaluation. We always know that all
suffixes stored between \texttt{suffixes[0]} and \texttt{suffixes[q-1]}, where
\(q=\texttt{left-suffixes}\), are already processed. If this range is large
enough to hold all suffixes between \texttt{left} and \texttt{right}, then we
can use the space to the left of \texttt{left} as space for \texttt{sbuffer}.
If this range is not large enough, we test if the number of elements in
\texttt{sbufferspace} is large enough. If not, then we enlarge it accordingly,
and return \texttt{sbufferspace}.\\


The function \texttt{getbufferspacelazy} does the space management for the
array \texttt{sbuffer} in the case of lazy evaluation. If the width of the
\texttt{sbufferspace} becomes smaller than some maximal width, then we reduce
the size of this array. If the array is too small, then we enlarge it
accordingly.\\


Since we have a virtual sentinel character which is different from all possible
256 characters of the ASCII-alphabet, a branching node can have
\texttt{UCHAR\_MAX+2=257} successor nodes. The sentinel edge leads to a leaf,
but each of the other edges can lead to a branching node. Hence, to evaluate a
branching node, we have to enlarge the size of \texttt{stree} by at least
\texttt{MAXSUCCSPACE} integers. This is done by the function
\texttt{allocstree}, when \texttt{next_free} reaches the end of the
block currently allocated for \texttt{stree}.


\section{Sorting and Grouping Suffixes}

To sort the suffixes of one group according to their first character,
we use the function \texttt{sortByChar} which implements a counting sort
algorithm. It takes as arguments the left and right bound \texttt{left} and
\texttt{right} of the subarray representing the group to be sorted, and
a commong prefix of all suffixes which has not been dropped from the
beginning of the suffix. In the first loop, this prefix is dropped
from each of the suffixes. Note that a bucket contains the suffixes
in the order of their length. This stability is achieved by inserting
them from right to left. Note that there is no loop over all the
characters in the alphabet. Instead to determine the boundaries
of the subgroups, one loops over all suffixes of the group. For small
groups this is faster.\\


We could also call \texttt{sortByChar} to determine the groups for all suffixes
of the input string. This would however require \texttt{sbuffer} to be of size
\texttt{textlen}. To save this space and to speed up the sorting, we have a
special version of \texttt{sortByChar} called \texttt{sortByChar0}. It scans
all suffixes from left to right, determines the size of each group, and then
directly sorts all suffixes into the array \texttt{suffixes}.\\


To determine the length of the longest common prefix of some suffixes, we
implement the function \texttt{grouplcp}. It loops over all \(j=1,2,\ldots\),
and tests for all suffixes if the \(j\)th characters are all identical. As soon
as a character is found which is not identical, \texttt{grouplcp} returns the
length \(j\) of the longest common prefix. Note that we already known that the
first character of all suffixes in the group are identical, so it is correct to
start with \(j=1\). As soon as the \(j\)th character of the smallest suffix of
the group is the sentinel character, the procedure stops and has found the
longest prefix.

\section{The Evaluation Process}

The function \texttt{evalsuccedges} evaluates all edges outgoing from an
unevaluated node \(\overline{u}\). The suffixes in the subtree below
\(\overline{u}\) are found between the pointers
\(\mathit{left}=\mathit{left}(\overline{u})\) and
\(\mathit{right}=\mathit{right}(\overline{u})\).  \texttt{evalsuccedges}
returns a reference to the first unevaluated branching node, if such a node
exists. Otherwise it returns \texttt{UNDEFREFERENCE}. \texttt{previousnode}
always refers to the left brother of the current node, if such a node exist.\\

The function \texttt{evalrootsuccedges} evaluates all edges outgoing from the
\(root\). It is a specialization of the previous function, and it additionally
initializes the array \texttt{rootchildtab} appropriately.\\

The function \texttt{evaluatenodeeager} evaluates the node referenced by
\texttt{node}. It is used for eager evaluation.  \texttt{evaluatenodeeager}
extracts the left and right boundaries of the group of suffixes which end in
the corresponding subtree. Then the \emph{lp}-value and \emph{firstchild}-value
of that node are set. Subsequently the longest common prefix of the suffixes of
that group is determined, and the group is divided into subgroups by sorting
these according to the first character of the remaining suffixes. Then the
edges outgoing from \texttt{node} are constructed.\\

The middle part of the function deals with the space management: In eager
evaluation, the suffix pointers in the array \texttt{suffixes} are processed
from left to right. In other words, if we have processed the suffixes up to the
suffix stored at, say \texttt{suffixes[q]}, then we do not need the entries at
indices \(0 \dots q\). Hence, if \(q\) reaches a certain threshold, we move all
suffix pointers at positions larger than \(q\) exactly \(q\) positions to the
left, and reduce the size of the array \texttt{suffixes} to the number of the
remaining suffixes. More precisely, we approximately halve the size of the
array \texttt{suffixes}. To have valid pointers into the array
\texttt{suffixes}, we let all suffix pointers be relative to the pointer
\texttt{suffixbase}. This is decremented by \(q\) as well.
\texttt{evaluatenodeeager} returns a reference to the next node to be processed
in eager evaluation.\\

The function \texttt{evaluatenodelazy} is identical to
\texttt{evaluatenodeeager}, except that it does not halve the size of the table
suffixes. This is not possible for lazy evaluation, since the suffixes are not
processed from left to right, but in an order determined by the traversal of
the unevaluated suffix tree. Unlike \texttt{evaluatenodeeager}, the function
does not return anything. \texttt{evaluatenodelazy} is used for lazy
evaluation.\\

Given a reference to a branching node, the function \texttt{getnextbranch}
returns a reference to the next branching node to the right. If there is no
such branching node, then \texttt{getnextbranch} returns
\texttt{UNDEFREFERENCE}.\\

For the eager evaluation of the suffix tree, we evaluate the nodes in depth
first and left to right order. To do so, we facilitate a stack, which stores
references to the branching nodes still to be evaluated. The stack is
implemented by an array, which is enlarged in units of 100 elements, if
necessary. The macros \texttt{NOTSTACKEMPTY}, \texttt{PUSHNODE},
\texttt{POPNODE} defines the API for the stack.\\

The function \texttt{evaluateeager} evaluates all nodes of the suffix tree in a
depth first left to right strategy. This is achieved by using a stack (rather
than recursively, which would use much more memory): In the outer loop the
elements are popped from the stack. In the inner loop we always test if the
current node has a right brother which is branching. If so, the right brother
is pushed, and we proceed with the first child of the current node which is
branching. This is of course unevaluated.\\

\section{Initializing the Data Structures}

The function \texttt{inittree} allocates and initializes the data structures.
In particular it inserts the root with depth 0, and the appropriate boundaries.
The latter are however not necessary, since if the depth is 0, we know them
anyway.

\section{Searching for Patterns}

The function lcp computes the length of the longest common prefix of two
strings given by pairs of pointers to the first and last characters.\\

We have two different functions to test whether some string occurs in the
suffix tree or not. One handles the case that the suffix tree is evaluated
lazily, the other handles the case that the suffix tree has been completely
evaluated before starting the pattern search. We only give the code for the
former. Both functions share a large amount of code. We therefore use three
macros: \texttt{CHECKROOTCHILD} tests for the given character
\texttt{firstchar}, whether there is a \texttt{firstchar}-edge outgoing from
the root. If this edge leads to a leaf, we test if the remaining pattern
suffix is a prefix of the label of the leaf edge. If the edge leads to a
branching node, then we continue with this.\\

The macro \texttt{CHECKLEAFEDGE} tests whether the remaining suffix of the
pattern is a prefix of the label of the current leaf edge.


The macro \texttt{CHECKBRANCHEDGE} tests whether the remaining suffix of the
pattern is a prefix of the label of the current edge to a branching node.

The function \texttt{firstchildlp} retrieves the \emph{lp}-value of the first child
for a given branching node. In case the firstchild is an unevaluated
branching node, the returned value is delivered by the macro
\texttt{GETLPUNEVAL}. In case the firstchild is a leaf or
an evaluated node, the returned value can directly be retrieved,
given the reference to the first child.

The function \texttt{occurslazy} takes a void pointer, as well as a pointer to
the input string and its length as an argument. These three parameters are not
used here, but we keep them since the function is passed to the function
\texttt{searchpattern} (defined in the library module \texttt{searchpat}). The
last two arguments are of interest. They refer to the first and the last
character of the pattern to be searched. The function works in two loops. The
outer loop scans down the suffix tree, and the inner loop scans the list of
successors of a given node. For the root there is no inner loop, since we
access table \texttt{rootchildtab}. Note that before following an edge to a
branching node, we first evaluate that node, if it is not already evaluated.\\


The function \texttt{occurseager} has the same interface as
\texttt{occurslazy}. It is slightly simpler, due to the fact that it assumes
that the suffix tree has been completely evaluated. We do not show the code
here.

\section{Putting it all together}

\texttt{wotd} implements the \emph{wotd}-algorithm and calls the functions to
search for patterns. If the first argument is true, then the suffix tree is
evaluated with the eager strategy. Otherwise, the suffix tree is evaluated with
the lazy strategy. Note that in this case the evaluation of nodes is done in
the function \texttt{occurslazy}.


\end{document}
